\documentclass[oribibl]{llncs}
\setcounter{secnumdepth}{3}

\begin{document}

\title{CM50200 Mobile and Pervasive Systems: 3. Interaction in and with the environment}


\author{Thomas Smith}
\institute{Centre for Digital Entertainment\\University of Bath\\\email{taes22@bath.ac.uk}}
\maketitle

\begin{abstract}
Lorum ipsum.
\end{abstract}

\section{Introduction}
% When we think of mobile and pervasive computing we often think of mobile phones, PDAs, tablet computers and the like. However there are many examples of interaction with the environment, in the environment, and where the computing is embedded in the environment. Give relevant examples of these, evaluate their success and consider what the design challenges are.

Device-free interaction in smart domestic environments \cite{Heidrich:2013:DIS:2459236.2459248}

%quoted abstract
In this keynote, I argue for a transition from designing Human-Computer Interaction to Human-Environment Interaction. This is done in the context of ambient intelligence and the disappearing computer, and the resulting challenges for designing interaction in future smart environments. Our approach is based on exploiting the affordances of real objects by augmenting their physical properties with the potential of computer-based support. Combining the best of both worlds requires an integration of real and virtual worlds resulting in hybrid worlds. In this approach, the computer "disappears" and is almost "invisible", but its functionality is ubiquitously available and provides new forms of interaction. The general comments are illustrated with examples from different projects. \cite{Streitz:2006:HIH:1783789.1783791}
%end quoted abstract

\subsection{Overview}
% Background defining the terms and the motivation of the paper. A number of relevant examples of these, evaluate their success and consider what the design challenges are.

\section{Background}
% Mobile and pervasive systems encompass more than just mobile devices. Define interaction with the environment, in the environment, and where the computing is embedded in the environment.

Active interaction, cf. Older user experience

\section{Existing Systems}	\label{sec:existing_systems}

Passive interaction, bluetooth device names to control and customise applications running in augmented everyday public spaces \cite{Davies:2009:UBD:1555816.1555832}.

Passive interaction, art in public spaces. Location/presence rather than deliberate intentional interaction \cite{Nguyen:2006:MSV:1180639.1180732}

Passive interaction with the environment TODO flesh out \cite{Vazquez:2004:IMP:1031419.1031437}

\subsection{Smart Environments}		\label{sub:smart_environments}
Augmentation

\subsubsection{Mixed Reality Environments}		\label{ssub:mixed_reality_environments}

% subsubsection mixed_reality_environments (end)
\subsubsection{Assistive Service Environments}		\label{ssub:assistive_service_environments}

Symbiosis: an innovative human-computer interaction environment for alzheimer's support \cite{Mandiliotis:2013:SIH:2526481.2526496} investigates differing roles within augmented environment; multimodal input spanning tradition interaction methods for caregivers (scheduling, forum etc) and a range of NUI approaches to (blah) for the patient. Kinect for guesture-based interaction, augmented reality with tagging for live reminders, and EEG for (look it up TODO).

% subsubsection assistive_service_environments (end)
% subsection smart_environments (end)

% section existing_systems (end)
\section{Proposed Systems}		\label{sec:proposed_systems}

% section proposed_systems (end)
\section{Design Challenges}		\label{sec:design_challenges}
\cite{Wiberg:2008:EIC:1543137.1543141}

A Smart Home Experience Using Egocentric Interaction Design Principles \cite{Surie:2012:SHE:2469451.2469943}

%quoted abstract
Context-aware intelligent environments are computing systems embedded within physical spaces. They are equipped with input and output computing devices for users and sensors to provide contextual information to the system. These environments provide new challenges to interface designers due to a number of differences from typical desktop computing environments, including the lack of a single focal point for the user, a dynamic set of interaction devices, the sensor-rich nature of the environment, the potential of multiple simultaneous users, and the opportunity for diverse interaction modalities. This essay describes these challenges and focuses on issues involving multiple interaction modalities and automatic system behaviors. \cite{Shafer:2001:IIC:1463108.1463124}
%end quoted abstract

Crowd interaction
%quoted abstract
Smart environments (e.g., airports, hospitals, stadiums, and other physical spaces using ubiquitous computing to empower many mobile people) provide novel challenges for usability engineers. Firstly, interaction can be implicit and therefore unintentional on the part of its users. Secondly, the impact of a smart environment can influence the collective or crowd behavior of those immersed within it. These challenges lead to requirements for complementary analyses which must be combined with the more typical focus on the interaction between user and device. The paper explores a family of stochastic models aimed at analyzing these features with a particular focus on crowd interaction.
\cite{Harrison:2009:ECI:1570433.1570456}
%end quoted abstract

\subsection{Interaction Methods}		\label{sub:interaction_methods}
Gestural
\cite{Carrino:2011:ATH:2027296.2027317}

Small lexicon of functional gestures \cite{Carrino:2013:FGH:2530824.2530844}
Spoken language for interaction with intellignet environments \cite{Minker:2009:RSL:1735821.1735825}
Tangible User Interface
\cite{Bartolini:2012:RNI:2387476.2387479}
Multimodal
\cite{Carrino:2011:HSE:2070481.2070501}
Multimodal interaction with a home operating system \cite{Weingarten:2010:TMI:1858171.1858255}


% subsection interaction_methods (end)

Augmenting everyday tables as interaction (pointing) devices \cite{Schmidt:2002:UIU:1765426.1765451}

how its application can enhance a post-WIMP human-environment interaction

% section design_challenges (end)
\section{Conclusion}

\newpage
\bibliographystyle{splncs03}
\bibliography{IitEreport}

\end{document}